%;; -*- mode:latex-mode -*-
% Author: M.Herhold
% Version: 0.3

\documentclass[oneside,a4paper,12pt]{article}
\thispagestyle{empty}

\usepackage{german}
\usepackage{changepage}
\usepackage[utf8]{inputenc}
\usepackage{tabularx} % verbesserte Tabellen
\usepackage[normalem]{ulem} % Text durchstreichen

\usepackage[default,scale=1.0]{opensans} %% Offizielle TU-DD Schriftart
\usepackage[T1]{fontenc}

% Einrückungen bei nächster Zeile verhindern
\parindent0pt
\setlength{\parindent}{0pt} 
\setlength{\parskip}{\baselineskip}

%%%%%%%%%%%
% folgende Zeilen bitte ausfüllen:
%

\def\Betreuer{Betreuernamen bitte hier eintragen!}
\def\Gruppennummer{ ??  }
\def\DatumDerDurchführung{Datum hier}

% durchgeführten Versuch bitte mit einem 'X' markieren
\def\VersuchTuersteuerung{X} %       ART-1
\def\VersuchEchtzeitsteuerung{ } %   ART-2
\def\VersuchFuellstandsteuerung{ } % ART-3

% bitte die Namen der Studenten der Gruppe eintragen (keine Matrikelnummern)
% Namensfeld einfach leer lassen, wenn weniger Mitglieder in der Gruppe sind.
\def\StudentEins{Bitte 1. Namen hier eintragen!}
\def\StudentZwei{Bitte 2. Namen hier eintragen!}
\def\StudentDrei{Bitte 3. Namen hier eintragen!}
\def\StudentVier{Bitte 4. Namen hier eintragen!}

%
% keine Änderungen ab hier notwendig
%%%%%%%%%%%

%
% change code below only if you know what you are doing!
% Feedback and code improvements are very welcome.
% mail to M.Herhold, TU-Dresden, Institute of Automation
%
\begin{document}

\begin{center}
    {\Huge \textbf{Praktikumsdokumentation}\\}
    \bigskip
    {\Large Mikrorechentechnik II}\\
\end{center}

\vfill

\renewcommand{\arraystretch}{1.2}
\begin{tabularx}{\textwidth}{p{2cm} p{0.3cm} p{1.5cm} X}
    \textbf{Versuch:} & \VersuchTuersteuerung & \ ART-1 & Türsteuerung\\ \cline{2-2}
    & \VersuchEchtzeitsteuerung & \ \sout{ART-2} & \sout{Echtzeitsteuerung}\\ \cline{2-2}
    & \VersuchFuellstandsteuerung & \ ART-3 & Füllstandsteuerung\\ \cline{2-2}
\end{tabularx}

\bigskip

\begin{tabularx}{\textwidth}{p{2cm} X}
  \textbf{Betreuer:} & \Betreuer \\ \cline{2-2}
\end{tabularx}

\begin{tabularx}{\textwidth}{p{2cm} p{0.38cm} X p{2.2cm}}
    \textbf{Gruppe:} & \Gruppennummer & \hfill \textbf{Datum} {\small der Praktikumsdurchführung}\textbf{:} & \DatumDerDurchführung \\
  \cline{2-2} \cline{4-4}
\end{tabularx}

\vfill

\renewcommand{\arraystretch}{3}
\rule{\textwidth}{0.5mm}
\begin{tabularx}{\textwidth}{p{4.2cm} | p{0.5cm} X p{0.5cm}}
  & & \StudentEins & \\ \cline{3-3}
  \textbf{Gruppenmitglieder:} & & \StudentZwei & \\  \cline{3-3}
  & & \StudentDrei & \\  \cline{3-3}
  & & \StudentVier & \\  \cline{3-3}
%  &\\
\end{tabularx}

\vfill
\renewcommand{\arraystretch}{1}
Informationen zur Abgabe des Protokolls:\\[0.4cm]
{\small
    \begin{tabularx}{\textwidth}{p{2.5cm} X}
      Termin:           & zwei Wochen nach Praktikumsdurchführung\\
      Art \& Weise:     & als PDF-Dokument per E-Mail an\\
                        & mario.herhold ( ä ) tu-dresden.de\\
    \end{tabularx}}

\end{document}

